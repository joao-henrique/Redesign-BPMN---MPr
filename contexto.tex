\section{Informações de contexto}

\subsection{Contexto}
	O cenário desse trabalho é o de uma empresa que
	fornece uma solução completa, composta por hardware e software.
	Essa solução tem como objetivo a emissão de cupons fiscais que são emitidos no
	caixa de supermercados durante a compra.
	O produto de software é responsável por identificar o item quando passados no
	identificador de código de barras. A solução de hardware é composta pelo identificador
	de código de barras, e o emissor de cupom fiscal. Como a empresa em questão é
	provedora de dois items, o serviço de atendimento ao
	usuário se faz necessário pois tais soluções podem apresentar algum defeito, ou
	o usuário não conseguir ter uma experiência de uso desejada.
	O público alvo dessa empresa são supermercados ou quaisquer empreendimento
	que busca ter catalogados seus items, e disponibilizar para venda. O contexto
	abordará somente o atendimento do usuário(supermercados) em relação ao uso das duas soluções.
	O processo de suporte aos usuários das soluções se encontra disforme e não vem obtendo
	os resultatos (relatos) esperados, causando insatisfação dos clientes, além de gastos por parte da empresa.
	Devido a esses problemas os usuários(supermercados) deixam de lucrar, agravando ainda mais a insatisfação
	dos clientes. Buscando resolver esse problema e ainda agregar valor para o usuário provendo
	um serviço eficiente e eficaz.
	\\A empresa solicitou que seu nome não fosse citado nos resultados aqui mostrados, esse pedido
	se fez necessário pois a mesma está sobre o processo de direito de imagem e venda.
