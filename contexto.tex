\section{Informações de contexto}

\subsection{Contexto}
	O cenário usado como referência nesse trabalho é o de uma empresa que
	fornece uma solução complete composta por  hardware e software.
	Essa solução tem como objetivo a emissão de cupons fiscais que são emitidos no
	caixa de supermercados durante a compra.
	O produto de software é responsável por identificar o item quando passados no
	identificador de código de barras. A solução de hardware é composta pelo identificador
	de código de barras, e emitir cupom fiscal.Como a empresa em questão é
	provedora de dois items, o serviço de atendimento ao
	usuário se faz necessário pois tais soluções podem apresentar algum defeito.
	O público alvo dessa empresa são supermercados ou quaisquer empreendimento
	que busca ter catalogados seus items e disponibilizar para venda. O contexto
	abordará somente o atendimento do usuário (supermercados) em relação ao uso das duas soluções.\\
	O processo de suporte aos usuários das soluções se encontra disforme e não vem obtendo
	os resultatos esperados, causando insatisfação dos clientes, além de gastos por parte da empresa.
	Devido a esses problemas os usuários(supermercados) deixam de lucrar, agravando ainda mais a satisfação
	dos clientes. Buscando resolver esse problema e ainda agregar valor para o usuário utilizando provendo
	um serviço eficiente e eficaz.
	Nos tópicos a seguir será mostrado os objetivos,fatores de sucesso, os processos
	identificados assim como uma analise do que foi encontrado

		A empresa solicitou que seu nome não fosse citado nos resultados aqui mostrados, esse pedido
	se fez necessário pois a mesma está sobre o processo de direito de imagem e venda.

 \subsection{Definição do Problema}
 \begin{itemize}[noitemsep]
   \item Problema:
     \begin{itemize}
       \item O processo de suporte ao usuário é demorado, e demanda de muitos recursos
 						humanos para que seja executado.
     \end{itemize}
   \item O problema afeta:
     \begin{itemize}
       \item Os clientes
     \end{itemize}
   \item Cujo impacto é:
     \begin{itemize}
       \item	O cliente que tem a demora da solucão do seu problema
 						deixa de lucrar, pois o equipamento normalmente fica sem uso.
     \end{itemize}
   \item Uma solução bem sucedida seria:
		 \begin{itemize}
       \item Uma boa solução seria um conjunto de atividades, que simplifique
       	automatize algumas atividades desse processo.
			\end{itemize}
 \end{itemize}

\subsection{Objetivos}
\begin{itemize}[noitemsep]
	\item Aumentar a satifsfação do cliente
  \item Formalizar o processo atual
  \item Diminuir a necessidade de tarefas humanas no processo de suporte
  \item Diminuição do tempo de processamento de uma requisição individual
	\item Projetar e implementar a sistema de monitoramento de perfomace
  \item Encontrar gargalos
\end{itemize}

\subsection{Principais indicadores}
\begin{itemize}[noitemsep]
  \item Avaliação dos usuários do sistema de suporte
	\item Tempo médio gasto para a realização de um atendimento
	\item Quantidade de recurso monetário gasto nesta macroatividade
\end{itemize}

\subsection{Envolvidos}
\begin{itemize}[noitemsep]
  \item Operador de suporte
  \item Desenvolvedor Sênior
  \item Representante de Vendas
  \item Chefe de tecnologia
\end{itemize}
