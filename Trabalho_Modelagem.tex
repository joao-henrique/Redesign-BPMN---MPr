\documentclass[11pt,a4paper]{article}
\usepackage[utf8]{inputenc}
\usepackage[T1]{fontenc}
\usepackage{blindtext}
\usepackage{enumitem}
\usepackage{hyperref}

\begin{document}


\begin{flushleft}
\section{Contexto}

\subsection{Motivação}
Quando se pensa sobre os principais fatores que colaboram para o sucesso de
empresas a  satisfação do cliente no mercado de TI exerce um papel importante,
e \textit{user experience} é com certeza um dos principais fatores, os caminhos
para prover a \textit{user experience} são muitos, por exemplo prover qualidade
no serviço de usuário, pode ser um desses caminhos. Consequentemente vemos o
suporte ao usuário como sendo um serviço provido por uma empresa  ao seus
clientes com o objetivo de melhorar a experiência com o produto provido. Em
outras palavras o serviço de suporte ao usuário ajuda ao cliente resolver
qualquer problema que possa encontrar enquanto usa o produto ou serviço

\end{flushleft}

\subsubsection{Serviço de suporte ao usuário}
Primeiramente devemos definir o que é o serviço de suporte na área de TI,
podemosencontrar vários termos como:
 		- Suporte tecnico
		- Service Desk
		- Help desk
		- Suporte ao Cliente
		- Suporte
		- Suporte ao Usuário
		- Etc
Basicamente esses termos definem a mesma coisa,mas cada um deles é focado
em diferentes aspectos do serviço, então para isso vamos focar no suporte
ao usuário, pois esse é mais comum e com isso evitamos ambiguidades
- Logo entende-se como suporte ao usuário um serviço provido por uma organização
para seus clientes para promover uma experiência com seu produto ou serviço,
resolvendo qualquer problema que o cliente possa encontrar enquanto usa o
serviço ou produto. Além disso não se  deve colocar qualquer restrição ao tipo
de problema que possa ser encontrado ou qualquer duvida ou denuncia que o
cliente possa reportar.
Como essa definição é bastante intuitiva nos podemos descrever o que
é serviço com os termos de service science


\subsection{Padrão industrial  e melhores praticas para suporte ao Usuário}
[TO DO]
\section{Contexto}
\subsection{Objetivos}
\begin{itemize}
  \item Formalizar o processo atual
  \item Encontrar gargalos
  \item Otimizar processos
\end{itemize}

\subsection{Fatores criticos de sucesso:}
\begin{itemize}
  \item Diminuição do tempo de processamento de uma requisição individual
  \item Diminuir a necessidade de tarefas humanas no processo de suporte
\end{itemize}

\subsection{Time do projeto}
\begin{itemize}
  \item Operador de suporte
  \item Desenvolvedor Sênior
  \item Representante de Vendas
  \item Chefe de tecnologia
\end{itemize}


\subsubsection{Planejamento do Projeto}
\begin{itemize}
\item Fase 1 - Analise \\
	- Formalização do processo corrente
	- Analise de informação do sistema de logs
\item Fase 2 - Otimização\\
	- Otimização de identificação dos processos\\
	- Documentação de novos processos
\item Fase 3 - Implantação \\
	- Implantação do processo\\
\item Fase 4 - Validação\\
	- Implantação do processo de monitoramento
\end{itemize}

\subsubsection{Fase 01}
~[TO DO - Diagramas]

\subsubsection{Conclusão e análise}
- Processos indefinidos informalmente\\
- Ausência de monitoramento\\
- Ausência de uma base de conhecimento\\
- Ausência de autênticação de usuário\\
- Processos de longo tempo de execução\\
- Reporte insuficiênte para o cliente\\

\subsection{Proposta de redesign}
- Gestão de Conhecimento e Auto-Ajuda

- Para base de conhecimento, será preciso incluir
	- Manual para suporte dos usuários
	- Documentação completa dos produtos de Software
	- Documentação de problemas comuns
	- Base de dados de erros e suas "workarounds"
	- Base de dados de "improvement sugestions"

A base de conhecimento será lançada como uma extensão do
suporte web page
- Monitoramento de performace

Como primeira ação foi necessário achar os indicadores chave de performace,
onde foi possivel mapear o relacionamento desses indicadores chave com os fatores
criticos de sucesso




\end{document}
